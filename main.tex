%%%%%%%%%%%%%%%%%%%%%%%%%%%%%%%%%%%%%%%%%%%%%%%%%%%%%%%%%
% This is a CV for Jim Tran Kelly based on the sample CV template created using altacv.cls (v1.1.4, 27 July, 2018) as written by LianTze Lim (liantze@gmail.com). This has been last updated by Jim Kelly, 8 December, 2020.
% This document may be distributed and or modified under the conditions of the LaTeX Project Public License, either version 1.3. The latest version of this license is available at http://www.latex-project.org/lppl.txt, and version 1.3 or later is considered a part of all distributions of LaTeX version 2003/12/01 or later.
%%%%%%%%%%%%%%%%%%%%%%%%%%%%%%%%%%%%%%%%%%%%%%%%%%%%%%%%%

% Command to pass options to 'xcolor'.
\PassOptionsToPackage{dvipsnames}{xcolor}

% If using \orcid or 'academicons', please ensure to enable the 'academicons' command here and then compile with XeLaTeX or LuaLaTeX;
% \documentclass[10pt,a4paper,academicons]{altacv}
% Note: Use the 'normalphoto' option if wanting to use a photo of standard dimensions rather than cropped to fit within a circle, i.e. the following command.
% \documentclass[10pt,a4paper,normalphoto]{altacv}

\documentclass[10pt, a4paper]{altacv}
% AltaCV uses the 'fontawesome' and 'academicon' fonts and packages. For more information view:
% texdoc.net/pkg/fontawecome and http://texdoc.net/pkg/academicons for a full list of symbols. For best results compile this document with LuaLaTeX. If wishing to compile this document with XeLaTeX, it is necessary to install Academicons.ttf in the given operating system's font folder.

% Define page layout
\geometry
	{
	left = 1cm,
	right = 9cm,
	marginparwidth = 6.8cm,
	marginparsep = 1.2cm,
	top = 1.25cm,
	bottom = 1.25cm,
	footskip = 2\baselineskip
	}

% Options for using pdflatex
\usepackage[T1]{fontenc}
\usepackage[utf8]{inputenc}
\usepackage[default]{lato}
\usepackage{setspace}

% If using XeLaTeX or LuaLaTeX then uncomment the following command
% \setmainfont{Lato}

% Define colour options for document
\definecolor{Navy}{HTML}{000080}
\definecolor{SlateGrey}{HTML}{2E2E2E}
\definecolor{LightGrey}{HTML}{666666}
\colorlet{heading}{Navy}
\colorlet{accent}{Navy}
\colorlet{emphasis}{SlateGrey}
\colorlet{body}{LightGrey}

% Define bullets for itemize and rating marker for \cvskill
\renewcommand{\itemmarker}{{\small\textbullet}}
\renewcommand{\ratingmarker}{\faCircle}

% The file, 'sample.bib' is used to store any publications.
\addbibresource{sample.bib}

\usepackage[colorlinks]{hyperref}

\begin{document}
    \singlespacing
    \name{Jim Tran Kelly}
    \tagline{Undergraduate Software Engineer}
    \personalinfo{
    % It must be noted that not all of these fields are necessary and it is possible to change & edit the sections 
    % through the following example command.
    % \printinfo{symbol}{detail}
    % Additionally, the 'academicons' option must be added to the \documentclass and then compiled with LuaLaTeX 
    % or XeLaTeX; if wanting to use \orcid or other 'academicons' commands.
        \email{jimkelly.t@outlook.com}
        \phone{+61412561162}
        \homepage{https://portfolio-website-76885.web.app}
        %\twitter{@twitterhandle}
        \linkedin{www.linkedin.com/in/jimkellyt}
        \github{https://github.com/jamestkelly}
    }
    % Extend the header to the full width of page.
    \begin{fullwidth}
    \makecvheader
    \end{fullwidth}
    % Note: For personal preference if wanting to make the font sizes of itemize environments smaller, it is possible to 
    % use the following example command.
    % \AtBeginEnvironment{itemize}{\small}
    \cvsection[page1sidebar]{Experience}
    % Enter experience fields
    	\cvevent{FutureNet}{Undergraduate Software Engineer}{November, 2020 -- Present}{West End, Brisbane}
        \begin{itemize}
        %% TO DO: Change the description for this position:
        %%		-> Expand on it more, i.e. specifics of what the role requires
        %%		-> Explain what technical aspects the role requires
        %%		-> Get confirmation from employer
        %%		-> Remove "assisted" specify what components I am working on exactly
        %%		-> Can condense down on the bar work positions as they provide little to no benefit for software jobs
        %%		-> Elaborate on which API it was (once completed)
        	   \item Worked within an international software development teams developing predominantly with TypeScript \& JavaScript.
            \item Set up work environments and built progressive web applications for use on desktop and mobile operating systems.
       \end{itemize}
        \divider
        \cvevent{Woolly Mammoth Alehouse}{Bartender}{Dec 2019 -- Mar 2020}{Fortitude Valley, Brisbane}
        \cvevent{}{Senior Bartender}{Jul 2018 -- Jan 2019}{}
        \cvevent{}{Floor Attendant (Glassy)}{Nov 2017 -- Jul 2018}{}
        \begin{itemize}
            \item Started at the Woolly Mammoth as a glassy and was promoted from glassy to a bar attendant and subsequently a supervisor position. Returned to the Woolly Mammoth Alehouse during the Summer holidays after a sustained break to focus on my studies.
        \end{itemize}
    	\divider
    	\cvevent{Ric's Bar}{Bartender}{Mar 2017 -- Nov 2017}{Fortitude Valley, Brisbane}
	\cvevent{}{Floor Attendant (Glassy)}{Nov 2016 -- Mar 2017}{}
    	\divider
    	\cvevent{The Morrison Hotel}{Food \& Beverage Attendant}{Feb 2016 -- Jul 2016}{Woolloongabba, Brisbane}
    \cvsection{Education}
    	\cvevent{Bachelor of Information Technology (Computer Science)}{Queensland University of Technology}{2018 -- Ongoing}{}
	\begin{itemize}
		\item Expected completion date: November 2021
		\item Coursework in Intelligent Systems (Artificial Intelligence) and Cyber Security
		\item Current Grade Point Average (GPA): 6.313 / 7.0
		\item Transitioned \& transferred units from a Bachelor of Electrical Engineering (Honours) to Bachelor of Information Technology (Computer Science).
	\end{itemize}
    \medskip
    \par\noindent\rule{\textwidth}{0.4pt}
    References provided upon request.
\end{document}